\documentclass[12pt]{article}
\usepackage{amsmath}
\usepackage{amsthm}
\usepackage{amsfonts}
\usepackage{amssymb}
\usepackage{authblk}
\usepackage{tkz-euclide}
\usepackage{tikz}
\usepackage{changepage}
\usepackage{lipsum}
\usepackage{tree-dvips}
\usepackage{qtree}
\usepackage[linguistics]{forest}
\usepackage[hidelinks]{hyperref}
\usepackage{mathtools}
\usepackage{blindtext}
% \usepackage[cal=esstix,frak=euler,scr=boondox,bb= pazo]{mathalfa}
\usepackage{graphicx}
\graphicspath{{./images/}}
\allowdisplaybreaks
\allowbreak
\theoremstyle{definition}
\newtheorem{definition}{Definition}
\newtheoremstyle{named}{}{}{\itshape}{}{\bfseries}{.}{.5em}{\thmnote{#3's }#1}
\theoremstyle{named}
\newtheorem*{namedconjecture}{Distinct Factorizations Conjecture}
\newtheorem{conjecture}{Conjecture}
\DeclareMathOperator{\sech}{sech}
\DeclareMathOperator{\arcsec}{arcsec}
\newcounter{customDef}
\renewcommand{\thecustomDef}{\arabic{customDef}}
\newcommand{\Mod}[1]{\ (\mathrm{mod}\ #1)}
\begin{document}
\title{Math 115 Definitions}
\author{}
\date{}
\maketitle
\date

\begin{enumerate}
    \item The product of sets $A \times B$ is the Cartesian product of the sets, where $A \times B = \{(a,b) \mid a \in A, b \in B\}$. 
    \item A relation on a set $A$ is a subset of $A \times A$. Elaborately, a relation on a set $A$ takes two values from $A$ and puts them into a class based on how they are compared. 
    \item A relation is reflexive if for any $a \in A$, $aRa$, symmetric if $aRb \implies bRa$, and transitive if $aRb \land bRc \implies aRc$, where $b,c \in A$. 
    \item A relation on a set $A$ is an equivalence relation if it is reflexive, symmetric, and transitive. An equivalence class of an element $a \in A$ is the set $\{x \in A \mid a \sim x\}$, which is the set of all members that are in $a$'s equivalence class.
    \item A partition of a set $A$ is a collection of disjoint subsets of $A$ (with each subset nonempty) such that their union is $A$. 
    \item $a$ and $b$ are relatively prime if $\gcd(a,b) = 1$. 
    \item The integers $b_1,\dots,b_n$ are relatively prime if $\gcd(b_1,\dots,b_n) = 1$. They are pairwise relatively prime if $\gcd(b_i,b_j) = 1$ for all $i \neq j$. 
    \item A prime number is an integer at least two whose factors are 1 and itself. A composite number is a number that isn't prime. 
    \item The prime factorization of a number $n$ is denoted $\prod_p p^{\alpha(p)}$, where this product symbolizes the product of all primes and the function $\alpha$ returns the exponent of a prime when considering that prime as its input.
    \item A congruence class (modulo $m$) is a set of all integers that are congruent modulo $m$. 
    \item A complete residue system (modulo $m$) is a set of integers $r_1,\dots,r_n$ such that any integer $x$ is congruent modulo $m$ to exactly of the $r_i$'s. 
    \item A reduced residue system (modulo $m$) is a set of integers $s_1,\dots,s_k$ coprime to $m$ such that any integer coprime to $m$ is congruent modulo $m$ to exactly one of the $s_i$'s. 
    \item Euler's totient function, $\phi(m)$, returns the number of elements in a reduced residue system modulo $m$. Equivalently, $\phi(m)$ is the number of integers $t$, with $0 < t \leq m$, such that $t$ is coprime to $m$. 
    \item Consider the integers modulo $m$. Then, take the integer $a$ in modulo $m$. Then, $a$ has a unique inverse (modulo $m$) $a^{-1}$ such that $aa^{-1} \equiv 1 \Mod{m}$. 
    \item A Gaussian integer is a complex number of the form $a + bi$, where $a,b \in \mathbb{Z}$. 
    \item $\mathbb{Z}[x]$ is the set of all polynomials with integer coefficients.
    \item The number of solutions to the congruence $f(x) \equiv g(x) \Mod{m}$ is the number of congruence classes that satisfy $f(x) - g(x) \equiv 0 \Mod{m}$. 
    \item Let $f(x) = a_nx^n + a_{n-1}x^{n-1} + \dots + a_0$, with $a_i \in \mathbb{Z}$ for all $i$. The degree of the congruence $f(x) \equiv 0 \Mod{m}$ is the highest value of $i$ such that $m \nmid a_i$, and undefined if $m \mid a_i$ for all $i$. To find the degree of the congruence $f(x) = g(x) \Mod{m}$ (with $f,g \in \mathbb{Z}[x]$), find the degree of $(f-g)(x) \equiv 0 \Mod{m}$. 
    \item A polynomial-time algorithm is an algorithm whose run time is a polynomial function of the length of its input.
    \item A weak probable prime to the base $a$ is a number $p>1$ that satisfies $a^{p-1} \equiv 1 \Mod{p}$. A weak pseudoprime to the base $a$ is a number $p>1$ that satisfies $a^{p-1} \equiv 1 \Mod{p}$ but $p$ is composite. 
    \item Consider the following algorithm: \\
        \begin{enumerate}
            \item Find $j$ and $d$ odd such that $m-1 = 2^jd$. 
            \item If $a^d \equiv \pm 1 \Mod{m}$, then $m$ is a strong probable prime, stop. 
            \item Square $a^d$ to get $a^{2d}$. If $a^{2d} \equiv 1 \Mod{m}$, then $m$ is composite. If $a^{2d} \equiv -1 \Mod{m}$, then $m$ is a strong probable prime, stop. 
            \item Repeat this procedure for the list $a^{4d}, \dots, a^{2^{j-1}d}$. 
            \item If the procedure has not yet terminated, $m$ is composite. 
        \end{enumerate}
    If the test is inconclusive, then $m$ is composite. $m$ is a strong pseudoprime to the base $a$ if the test with $m$ is conclusive but $m$ is both odd and composite. 
    \item A Carmichael number is a composite number $m$ which is a weak pseudoprime to the base $a$ for all integers $a$ coprime to $m$. 
    \item $p^\alpha$ exactly divides $n$ (denote: $p^\alpha \mid\mid n$) if $p^\alpha \mid n$ but $p^{\alpha + 1} \nmid n$. \\
    % the following 3 lines represent a horizontal bar 
    \begin{center}
        \hrule
    \end{center}
    \item \textbf{Root of $f \in \mathbb{Z}[x]$ modulo $m$. } Let $f \in \mathbb{Z}[x]$ and let $m \in \mathbb{Z}_{>0}$. Then, a root of $f$ modulo $m$ is an integer $a$ such that $f(a) \equiv 0 \Mod{m}$. 
    \item \textbf{Monic Polynomial. } A polynomial in $\mathbb{C}[x]$ (or $\mathbb{Z}[x])$ is monic if (it is nonzero) its leading coefficient is 1. 
    \item \textbf{$\mathbb{Z}/m\mathbb{Z}$. } This is the set of congruence classes modulo $m$. 
    \item \textbf{$\left(\mathbb{Z}/m\mathbb{Z}\right)^\ast$} This set is defined to be the set $\{\tilde{a} \in \mathbb{Z} : \gcd(a,m) = 1\}$. This set is well-defined and contains $\phi(m)$ elements. 
    \item \textbf{Order of $a$ modulo $m$. } Let $a \in \mathbb{Z}$ and $m \in \mathbb{Z}_{>0}$ with $\gcd(a,m) = 1$. Then the order of $a$ modulo $m$ is the smallest integer $h > 0$ such that $a^h \equiv 1 \Mod{m}$. If $\gcd(a,m) \neq 1$, then the order of $a$ modulo $m$ is \underline{undefined}. 
    \item \textbf{Primitive root modulo $m$. } A primitive root modulo $m$ is an integer $g$ whose order modulo $m$ is $\phi(m)$. 
    \item \textbf{Quadratic residue modulo $m$. } Let $m \in \mathbb{Z}_{>0}$. A quadratic residue modulo $m$ is an integer $a$ coprime to $m$ such that $x^2 \equiv a \Mod{m}$ has a solution. 
    \item \textbf{Quadratic non-residue modulo $m$. } Let $m \in \mathbb{Z}_{>0}$. A quadratic non-residue modulo $m$ is an integer $a$ coprime to $m$ such that $x^2 \equiv a \Mod{m}$ does not have a solution. 
    \item \textbf{Legendre Symbol, $\left(\frac{a}{p}\right)$. } Let $a \in \mathbb{Z}$. Then, $\left(\frac{a}{p}\right)$ is defined to be 1 if $a$ is a quadratic residue modulo $p$, -1 if $a$ is a quadratic non-residue modulo $p$, and 0 if $p \mid a$. 
    \item \textbf{Jacobi Symbol, $\left(\frac{P}{Q}\right)$. } Let $Q$ be an odd positive integer with prime factors $Q = q_1 \cdot \dots \cdot q_s$, where all the $q_i$ are odd primes, not necessarily distinct. Then, the Jacobi Symbol $\left(\frac{P}{Q}\right)$ is defined by: 
    $$
    \left(\frac{P}{Q}\right) = \prod_{j=1}^s \left(\frac{P}{q_j}\right)
    $$
    where $\left(\frac{P}{q_j}\right)$ is the Legendre Symbol. 
    \item \textbf{Binary Quadratic Form. } A binary quadratic form is a polynomial of the form $ax^2 + bxy + cy^2$, where $x,y$ are the variables, and $a,b,c$ are the coefficients. 
    \item \textbf{Binary Quadratic Form represents $n$. } A binary quadratic form $f=f(x,y)$ represents an integer $n$ if $f(x_0,y_0) = n$ for some $x_0,y_0 \in \mathbb{Z}$ with $(x_0,y_0) \neq (0,0)$. 
    \item \textbf{Binary Quadratic Form properly represents $n$. } The binary quadratic form $f=f(x_0,y_0)$ properly represents $n$ if $f(x_0,y_0) = n$ with $x_0,y_0 \in \mathbb{Z}$ relatively prime. 
    \item \textbf{Discriminant of a binary quadratic form. } The discriminant of a binary quadratic form $ax^2 + bxy + cy^2$ is the quantity $d = b^2 - 4ac$. 
    \item \textbf{Types of binary quadratic forms. } A binary quadratic form $f(x,y)$ is: 
    \begin{enumerate}
        \item indefinite if it takes on both positive and negative values. 
        \item positive semidefinite if $f(x_0,y_0) \geq 0$ for all $x_0,y_0$. 
        \item positive definite if $f(x_0,y_0) > 0$ for all $x_0,y_0$ with $(x_0,y_0) \neq (0,0)$. 
        \item negative semidefinite if $f(x_0,y_0) \leq 0$ for all $x_0,y_0$. 
        \item negative definite if $f(x_0,y_0) < 0$ for all $x_0,y_0$ with $(x_0,y_0) \neq (0,0)$. 
        \item definite if it is positive definite or negative definite. 
        \item semidefinite if positive semidefinite or negative semidefinite. 
    \end{enumerate}
    (note: we let $x_0,y_0 \in \mathbb{R},\mathbb{Q}, \mathbb{Z}$). 
    \item \textbf{$T_M$, where $M$ is a 2x2 matrix. } Given a 2x2 matrix $M$, let $T_M: \mathbb{R}^2 \to \mathbb{R}^2$ be the function $\begin{pmatrix}
        x \\
        y
    \end{pmatrix} \mapsto M\begin{pmatrix}
    x \\
    y
    \end{pmatrix}$. In other words, if $M = \begin{pmatrix}
        a & b \\
        c & d
    \end{pmatrix}$, then $T_M \begin{pmatrix}
        x \\
        y
    \end{pmatrix} = (ax+by, cx+dy)$. 
    \item \textbf{Modular group, $\Gamma$. } The modular group $\Gamma$ is the set $\Gamma = $ \{2x2 matrices with integer entries and determinant 1\}. 
    \item \textbf{Determinant of a matrix $M$. } If $M = \begin{pmatrix}
        a & b \\
        c & d
    \end{pmatrix}$, then the determinant of $M$ is given by $\det M = ad - bc$. 
    \item \textbf{$M$ takes $f$ to $g$. } A matrix $M \in \Gamma$ takes $f$ to $g$ (where $f$ and $g$ are forms) if $f \circ T_M = g$. 
    \item \textbf{Equivalent Forms. } The forms $f$ and $g$ are equivalent if $\exists M \in \Gamma$ that takes $f$ to $g$. 
    \item \textbf{Reduced binary quadratic form. } Let $f(x,y) = ax^2 + bxy + cy^2$ be a form whose discriminant is not a perfect square. Then, $f$ is reduced if $-|a| < b \leq |a| < |c|$ or $0 \leq b \leq |a| = |c|$. 
    \begin{center}
        \hrule
    \end{center}
    \item \textbf{Class number of $d$. } Let $d \in \mathbb{Z}$, not a perfect square. Then the class number of $d$ is the number of equivalence classes of forms of discriminant $d$, excluding classes of negative definite forms (if $d<0$). The class number is denoted $H(d)$. 
    \item \textbf{Automorph of $f$. } Let $f$ be a positive definite form. Then, an automorph of $f$ is a matrix $M \in \Gamma$ that takes $f$ to itself. 
    \item \textbf{$w(f)$}. The number of automorphs of a positive definite form $f$ is written as $w(f)$. 
    \item \textbf{Arithmetic Function. } An arithmetic function is a function $f: \mathbb{Z}_{>0} \to \mathbb{C}$. 
    \item \textbf{Multiplicative Function. } A multiplicative function is an arithmetic function (not the zero function) with $f(mn) = f(m)f(n)$ for all coprime $m,n \in \mathbb{Z}_{>0}$. 
    \item \textbf{Totally Multiplicative Function. } A totally multiplicative function is an arithmetic function, not the zero function, $f(mn) = f(m)f(n)$ for all $m,n \in \mathbb{Z}_{>0}$. 
    \item \textbf{$d(n), \sigma(n), \sigma_k(n), \omega(n), \Omega(n)$. } Let $n \in \mathbb{Z}_{>0}$. Then define: 
    \begin{enumerate}
        \item $d(n)$ to be the number of positive divisors of $n$. 
        \item $\sigma(n)$ to be the sum of the positive divisors of $n$. 
        \item $\sigma_k(n)$ to be the sum of the $k^{th}$ powers of the positive divisors of $n$. 
        \item $\omega(n)$ to be the number of distinct primes dividing $n$. 
        \item $\Omega(n)$ to be the number of primes dividing $n$, counting multiplicity. 
    \end{enumerate}
    \item \textbf{Möbius Function, $\mu$. } The Möbius function is the arithmetic function $\mu: \mathbb{Z}_{>0} \to \mathbb{Z}$ defined by $\mu(n) = (-1)^{\omega(n)}$ if $n$ is square-free and $\mu(n) = 0$ if otherwise. 
    \item \textbf{Finite Continued Fraction. } A finite continued fraction is something of the form: 
    $$
    a_0 + \frac{1}{a_1 + \frac{1}{a_2 + \frac{1}{\ddots + \frac{1}{a_n}}}}. 
    $$
    \item \textbf{Finite Simple Continued Fraction. } Let $\langle x_0,\dots,x_n \rangle$ be a finite continued fraction. It is a finite simple continued fraction if $x_0,\dots,x_n \in \mathbb{Z}$. 
    \item \textbf{Infinite Simple Continued Fraction. } An infinite simple continued fractions is an expansion $\langle a_0,a_1,\dots \rangle$ with $a_i \in \mathbb{Z}$ for all $i \geq 0$ and $a_i > 0$ for all $i > 0$. 
    \item \textbf{Value of an Infinite Simple Continued Fraction. } The value of an infinite simple continued fraction is $\lim_{n \to \infty} \langle a_0,\dots,a_n \rangle$. 
    \item \textbf{$n^{th}$ convergent of infinite simple continued fraction. } This is defined to be $r_n = \langle a_0,\dots,a_n \rangle = \frac{h_n}{k_n}$. 
    \item \textbf{Fractional Part $\{x\}$ of $x \in \mathbb{R}$. } Let $x \in \mathbb{R}$. The fractional part of $x$ is defined to be $x - [x]$, where $[]$ is the greatest-integer function. 
    \item \textbf{Distance $||x||$ from $x \in \mathbb{R}$ to the nearest integer. } Let $x \in \mathbb{R}$. Then, this quantity is defined to be $||x|| = \min \{\{-x\}, \{x\}\} \in [0,\frac{1}{2}]$. 
    \item \textbf{Periodic Continued Fraction. } An infinite simple continued fraction $\langle a_0,a_1,\dots \rangle$ is periodic if there is an integer $n>0$ such that $a_{r+n} = a_r$ for all sufficiently large $r$. 
    \item \textbf{Purely Periodic Continued Fraction. } An infinite simple continued fraction $\langle a_0,a_1,\dots \rangle$ is purely periodic if $a_{r+n} = a_r$ for all $r \geq 0$. 
    \item \textbf{Quadratic Irrational. } A quadratic irrational is a real number which is irrational which is a zero of a quadratic polynomial (nonzero) with integer coefficients. 
    \item \textbf{Conjugate of Quadratic Irrational. } Let $d \in \mathbb{Z}$, not a perfect square. THen, the conjugate of a quadratic irrational $r + s\sqrt{d}$ (with $r,s \in \mathbb{Q}$) is $r - s\sqrt{d}$. 
    \item \textbf{Pell's Equation. } This is the equation $x^2 - dy^2 = N$, where $d,N \in \mathbb{Z}$ and we are looking for solutions in integers $x,y$. 
    \item \textbf{Positive Solution of Pell's Equation. } A positive solution of Pell's equation is one where $x>0$ and $y>0$. 
\end{enumerate}


\end{document}